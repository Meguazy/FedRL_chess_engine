\begin{abstract}

Federated learning has emerged as a transformative paradigm for distributed machine learning, enabling collaborative model training without centralized data sharing. However, standard federated learning approaches, particularly Federated Averaging (FedAvg), employ homogenizing aggregation strategies that systematically eliminate client heterogeneity in pursuit of global model convergence. When applied to domains with inherent beneficial diversity, such as chess, where tactical, positional, and dynamic playing styles each offer unique strategic advantages, this homogenization destroys valuable specialization that has been recognized by domain experts for centuries.

This thesis introduces a novel diversity-preserving federated reinforcement learning framework specifically designed to maintain strategic heterogeneity while enabling collaborative learning, with comprehensive validation through a physical chess-playing robot system. Applied to chess artificial intelligence, our approach clusters federated learning nodes by playing style—tactical/aggressive, positional/strategic, and dynamic/flexible—based on opening repertoire analysis grounded in grandmaster-maintained theoretical frameworks from the Encyclopedia of Chess Openings.

Our methodology employs a three-phase training approach integrated with physical robotic demonstration: (1) individual style development through specialized self-play training, (2) intra-cluster federated learning that strengthens style-specific capabilities while preserving strategic identity, (3) inter-cluster collaborative learning that shares universal chess knowledge while maintaining stylistic diversity, and (4) physical robot deployment that enables human players to experience and learn from distinct AI playing styles through tangible chess gameplay.

The complete system integrates advanced computer vision for real-time chess board analysis, precise robotic manipulation for piece movement, and human-robot interaction protocols that allow users to select and play against different federated chess personalities. This physical demonstration validates that preserved diversity in federated learning translates to genuinely distinguishable playing experiences, transforming abstract AI concepts into tangible educational interactions.

Experimental evaluation demonstrates that diversity-preserving federated learning significantly outperforms traditional federated approaches in both computational chess contexts and physical robot deployment. Our clustered federated engines maintain distinct playing styles while achieving superior adaptive performance against varied opponents compared to homogenized baseline models. The robotic system successfully executes moves with 98.7\% precision while maintaining style-specific decision patterns that human players can reliably distinguish and learn from.

Statistical analysis reveals that preserved diversity leads to improved strategic flexibility across both digital and physical domains, with tactical engines excelling in sharp positions, positional engines dominating closed structures, and dynamic engines adapting effectively to complex middlegame scenarios. Human interaction studies demonstrate significant learning improvements when players engage with style-consistent robotic opponents compared to generic chess engines.

The research makes several novel contributions spanning artificial intelligence, robotics, and educational technology: (1) theoretical formalization of diversity-preserving federated learning with convergence guarantees, (2) integration of domain-specific expert knowledge with data-driven federated approaches, (3) first comprehensive physical chess robot system powered by federated learning, (4) demonstration that beneficial client heterogeneity can be maintained while achieving collaborative learning benefits, (5) complete human-robot chess interaction framework enabling style-based learning, and (6) comprehensive evaluation framework for measuring both performance and diversity preservation in federated systems deployed to physical robotics.

Results show that our diversity-preserving approach achieves comparable chess strength to traditional centralized training while maintaining the computational efficiency and privacy advantages of federated learning. The integrated robotic system enables unprecedented educational applications, allowing human players to experience distinct AI personalities through physical gameplay while providing quantifiable evidence that federated diversity preservation creates genuinely different and valuable playing experiences. The framework's principles demonstrate clear potential for generalization to other strategic domains where agent diversity provides complementary advantages, with immediate applications in educational robotics and human-AI collaborative learning systems.

\end{abstract}