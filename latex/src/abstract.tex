\begin{abstract}

Federated learning has emerged as a transformative paradigm for distributed machine learning, enabling collaborative model training without centralized data sharing. However, standard federated learning approaches, particularly Federated Averaging (FedAvg), employ homogenizing aggregation strategies that systematically eliminate client heterogeneity in pursuit of global model convergence. When applied to domains with inherent beneficial diversity, such as chess, where tactical, positional, and dynamic playing styles each offer unique strategic advantages, this homogenization destroys valuable specialization that has been recognized by domain experts for centuries.

This thesis introduces a novel diversity-preserving federated reinforcement learning framework specifically designed to maintain strategic heterogeneity while enabling collaborative learning. Applied to chess artificial intelligence, our approach clusters federated learning nodes by playing style, tactical/aggressive, positional/strategic, and dynamic/flexible, based on opening repertoire analysis grounded in grandmaster-maintained theoretical frameworks from the Encyclopedia of Chess Openings.

Our methodology employs a three-phase training approach: (1) individual style development through specialized self-play training, (2) intra-cluster federated learning that strengthens style-specific capabilities while preserving strategic identity, and (3) inter-cluster collaborative learning that shares universal chess knowledge while maintaining stylistic diversity. The approach leverages network analysis of chess opening relationships, validated through empirical studies of over 472,000 online games, to establish natural clustering boundaries that preserve beneficial heterogeneity.

Experimental evaluation demonstrates that diversity-preserving federated learning significantly outperforms traditional federated approaches in chess contexts. Our clustered federated engines maintain distinct playing styles while achieving superior adaptive performance against varied opponents compared to homogenized baseline models. Statistical analysis reveals that preserved diversity leads to improved strategic flexibility, with tactical engines excelling in sharp positions, positional engines dominating closed structures, and dynamic engines adapting effectively to complex middlegame scenarios.

The research makes several novel contributions: (1) theoretical formalization of diversity-preserving federated learning with convergence guarantees, (2) integration of domain-specific expert knowledge with data-driven federated approaches, (3) demonstration that beneficial client heterogeneity can be maintained while achieving collaborative learning benefits, and (4) comprehensive evaluation framework for measuring both performance and diversity preservation in federated systems.

Results show that our diversity-preserving approach achieves comparable chess strength to traditional centralized training while maintaining the computational efficiency and privacy advantages of federated learning. Furthermore, the preserved stylistic diversity enables superior performance against human players and varied AI opponents, validating the importance of maintaining beneficial heterogeneity in collaborative learning systems. The framework's principles demonstrate potential for generalization to other strategic domains where agent diversity provides complementary advantages.

\end{abstract}