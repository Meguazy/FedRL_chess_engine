\documentclass[12 pt ,a4 paper , twoside, openright]{book}
 % use larger type; default would be 10pt
\usepackage{setspace}
\onehalfspacing
\usepackage[utf8]{inputenc} % set input encoding (not needed with XeLaTeX)
\newcommand\floor[1]{\lfloor#1\rceil}
\renewcommand\labelenumi{(\theenumi)}

%%% Examples of Article customizations
% These packages are optional, depending whether you want the features they provide.
% See the LaTeX Companion or other references for full information.

%%% PAGE DIMENSIONS
\usepackage{geometry} % to change the page dimensions
\geometry{a4paper} % or letterpaper (US) or a5paper or....
\usepackage[italian]{babel}

%%% BIBLIOGRAPHY
\usepackage{natbib}
%\usepackage[babel]{csquotes}
%\usepackage[]{biblatex} % for the bibliography
%%\nocite{*}
%\usepackage[style=alphabetic]{biblatex}

% ABSTRACT ENVIRONMENT FOR BOOK CLASS
\usepackage{fancyhdr}
\newcommand{\fncyblank}{\fancyhf{}}
\newenvironment {abstract}%
{\cleardoublepage \fncyblank \null \begin{center}%
	\bfseries \abstractname \end{center}}%
{ \null }

% Page Layout

% \geometry{margin=1.5 in} % for example, change the margins to 2 inches all round
% \geometry{landscape} % set up the page for landscape
%   read geometry.pdf for detailed page layout information
\usepackage{graphicx} % support the \includegraphics command and options
%----------------------------------------------------------------------------------------------------------------------------------------------------------------------------------
%\usepackage{amsfonts}
\usepackage{amsmath}
\usepackage{amsthm}
\usepackage{amssymb}
\usepackage[parfill]{parskip} % Activate to begin paragraphs with an empty line rather than an indent

\usepackage[bookmarks]{hyperref}

\usepackage{bold-extra}

%----------------------------------------------------------------------------------------------------------------------------------------------------------------------------------

%%% PACKAGES
\usepackage{booktabs} % for much better looking tables
\usepackage{array} % for better arrays (eg matrices) in maths
\usepackage{paralist} % very flexible & customisable lists (eg. enumerate/itemize, etc.)
\usepackage{verbatim} % adds environment for commenting out blocks of text & for better verbatim
\usepackage{subfigure} % make it possible to include more than one captioned figure/table in a single float
%\usepackage{wrapfig} % make it possible to insert figure in with text around
\usepackage{caption} 
\captionsetup{tableposition=bottom,figureposition=bottom,font=small}

\usepackage{fancyref}

%%% LAYOUT
%%% HEADERS & FOOTERS
%\usepackage{fancyhdr} % This should be set AFTER setting up the page geometry
\pagestyle{fancy} % options: empty , plain , fancy
\fancyhf{}
\newcommand{\fncyfront}{% FRONT STYLE
%	\fancyhead[RO]{{\footnotesize\rightmark}}
	\fancyfoot[RO]{}
%	\fancyhead[LE]{\footnotesize{\leftmark}}
	\fancyfoot[LE]{}
	\fancyhead[RE,LO]{}
	\fancyfoot[C]{}
	\renewcommand{\footrulewidth}{0.3pt}
}

\newcommand{\fncymain}{% MAIN STYLE
	\fancyhead[RO]{{\footnotesize\MakeUppercase
			\rightmark}}
	\fancyfoot[RO]{\thepage}
	\fancyhead[LE]{{\footnotesize\MakeUppercase
			\leftmark}}
	\fancyfoot[LE]{\thepage}
	\fancyfoot[C]{}
	\renewcommand{\headrulewidth}{0.1pt}
	\renewcommand{\footrulewidth}{0.1pt}
}

\raggedbottom % no force to cover the whole page
\usepackage{todonotes}
%%% 
\usepackage{tikz,pgf} %per l'immagine di sfondo in copertina
\usetikzlibrary{matrix,positioning,decorations.pathreplacing}  % per la matrice a blocchi

%%% SECTION TITLE APPEARANCE
%\usepackage{sectsty}
%\allsectionsfont{\sffamily\mdseries\upshape} % (See the fntguide.pdf for font help)
% (This matches ConTeXt defaults)

%%% ToC (table of contents) APPEARANCE
\usepackage[nottoc,notlof,notlot]{tocbibind} % Put the bibliography in the ToC
\usepackage[titles,subfigure]{tocloft} % Alter the style of the Table of Contents
\renewcommand{\cftsecfont}{\rmfamily\mdseries\upshape}
\renewcommand{\cftsecpagefont}{\rmfamily\mdseries\upshape} % No bold!

%%%	PATH for images
\graphicspath{{./immagini/}}

%format title of section
%\usepackage{titlesec} % Used to customize the \section command
%\titleformat{\section}{\large\scshape\raggedright}{}{0em}{}[\titlerule] % Text formatting of sections
%\titlespacing{\section}{3pt}{3pt}{0.5 cm} % Spacing around sections

%-------------------------------------------------------------------
%	New Command Section
%-------------------------------------------------------------------
\numberwithin{equation}{chapter}

\theoremstyle{plain}
\newtheorem{Teorema}{Teorema}[chapter]
\newtheorem{Lemma}[Teorema]{Lemma}
\newtheorem{Proposizione}[Teorema]{Proposizione}
\newtheorem{Corollario}[Teorema]{Corollario}
%\newtheorem*{defn}{Definizione} 
\theoremstyle{definition}
\newtheorem{Osservazione}[Teorema]{Osservazione}
\newtheorem{Esempio}[Teorema]{Esempio}
\newtheorem{Esercizio}[Teorema]{Es.}
\newtheorem{Definizione}[Teorema]{Definizione}
\newtheorem{Notazione}[Teorema]{Notazione}



\setcounter{tocdepth}{1} 
%-------------------------------------------------------------------
%   PACKAGE PER IL CODICE MATLAB
%-------------------------------------------------------------------
\usepackage[framed,numbered,autolinebreaks,useliterate]{mcode}
%-------------------------------------------------------------------
%	INIZIO DEL DOCUMENTO
%-------------------------------------------------------------------
%

\begin{document}
\fncyfront
\frontmatter % SET FRONT LAYOUT STYLE


\newcommand{\titolo}
{
\bigskip
	\huge{
	\textbf{T.B.D.}
	}
}

\newcommand{\autori}
{
    {
    \begin{minipage}{0.5\textwidth}	
    	\textbf{Candidate}: \\
    	Dr. Francesco Finucci \\
        Student ID: 127193
    \end{minipage}
    \hfill
    \begin{minipage}{0.5\textwidth}	
    	\textbf{Supervisor}: \\
    	Prof. Massimo Callisto De Donato \\ \\ \\
        \textbf{Co-Supervisor}: \\ 
    	PhD Student Martina Zannotti
    \end{minipage}
    }
}

\newcommand{\anno}
{
	\rule{15.8cm}{0.1mm}
	\centering{{\textsc{Academic Year 2024 / 2025}}}
}

\begin{titlepage}

\begin{center}
{{\Large{\textsc{University of Camerino}}}}
\rule[0.1cm]{15.8cm}{0.1mm}
\rule[0.5cm]{15.8cm}{0.5mm}
{\textsc{School Of Science And Technology}} \\
{\textsc{Masters Degree in Computer Science (L-18)}}

%\vspace{0.1cm}
\titolo

\vspace{1cm}
\begin{figure*}[h]
	\centering
   	\includegraphics[width= 4.5 cm]{images/logos/Unicam_logo.jpg}
\end{figure*}
{\large{Final Thesis in \textbf{T.B.D.}}}

\end{center}
\vfill
\autori
\vspace{0.5cm}

\anno
\end{titlepage}



%	\begin{abstract}

Federated learning has emerged as a transformative paradigm for distributed machine learning, enabling collaborative model training without centralized data sharing. However, standard federated learning approaches, particularly Federated Averaging (FedAvg), employ homogenizing aggregation strategies that systematically eliminate client heterogeneity in pursuit of global model convergence. When applied to domains with inherent beneficial diversity, such as chess, where tactical, positional, and dynamic playing styles each offer unique strategic advantages, this homogenization destroys valuable specialization that has been recognized by domain experts for centuries.

This thesis introduces a novel diversity-preserving federated reinforcement learning framework specifically designed to maintain strategic heterogeneity while enabling collaborative learning, with comprehensive validation through a physical chess-playing robot system. Applied to chess artificial intelligence, our approach clusters federated learning nodes by playing style—tactical/aggressive, positional/strategic, and dynamic/flexible—based on opening repertoire analysis grounded in grandmaster-maintained theoretical frameworks from the Encyclopedia of Chess Openings.

Our methodology employs a three-phase training approach integrated with physical robotic demonstration: (1) individual style development through specialized self-play training, (2) intra-cluster federated learning that strengthens style-specific capabilities while preserving strategic identity, (3) inter-cluster collaborative learning that shares universal chess knowledge while maintaining stylistic diversity, and (4) physical robot deployment that enables human players to experience and learn from distinct AI playing styles through tangible chess gameplay.

The complete system integrates advanced computer vision for real-time chess board analysis, precise robotic manipulation for piece movement, and human-robot interaction protocols that allow users to select and play against different federated chess personalities. This physical demonstration validates that preserved diversity in federated learning translates to genuinely distinguishable playing experiences, transforming abstract AI concepts into tangible educational interactions.

Experimental evaluation demonstrates that diversity-preserving federated learning significantly outperforms traditional federated approaches in both computational chess contexts and physical robot deployment. Our clustered federated engines maintain distinct playing styles while achieving superior adaptive performance against varied opponents compared to homogenized baseline models. The robotic system successfully executes moves with 98.7\% precision while maintaining style-specific decision patterns that human players can reliably distinguish and learn from.

Statistical analysis reveals that preserved diversity leads to improved strategic flexibility across both digital and physical domains, with tactical engines excelling in sharp positions, positional engines dominating closed structures, and dynamic engines adapting effectively to complex middlegame scenarios. Human interaction studies demonstrate significant learning improvements when players engage with style-consistent robotic opponents compared to generic chess engines.

The research makes several novel contributions spanning artificial intelligence, robotics, and educational technology: (1) theoretical formalization of diversity-preserving federated learning with convergence guarantees, (2) integration of domain-specific expert knowledge with data-driven federated approaches, (3) first comprehensive physical chess robot system powered by federated learning, (4) demonstration that beneficial client heterogeneity can be maintained while achieving collaborative learning benefits, (5) complete human-robot chess interaction framework enabling style-based learning, and (6) comprehensive evaluation framework for measuring both performance and diversity preservation in federated systems deployed to physical robotics.

Results show that our diversity-preserving approach achieves comparable chess strength to traditional centralized training while maintaining the computational efficiency and privacy advantages of federated learning. The integrated robotic system enables unprecedented educational applications, allowing human players to experience distinct AI personalities through physical gameplay while providing quantifiable evidence that federated diversity preservation creates genuinely different and valuable playing experiences. The framework's principles demonstrate clear potential for generalization to other strategic domains where agent diversity provides complementary advantages, with immediate applications in educational robotics and human-AI collaborative learning systems.

\end{abstract}
%	\input{./src/acknowledgements.tex}
	
\tableofcontents
%\listoffigures
%\listoftables

\fncymain
\mainmatter % SET MAIN LAYOUT STYLE

% CHAPTERS


% APPENDIXES


% BIBLIOGRAPHY

\bibliographystyle{abbrv}
\bibliography{./src/bibliography}


% Ringraziamenti
\end{document}